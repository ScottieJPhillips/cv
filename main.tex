%______________________________________________________________________________________________________________________
% @briefLaTeX2e Resume for Kamil K Wojcicki
\documentclass[margin,line]{resume}
\usepackage[colorlinks = true,
linkcolor = blue,
urlcolor  = blue,
citecolor = blue,
anchorcolor = blue]{hyperref}

\usepackage{fontawesome}

% bibliography with mutiple entries
\usepackage{multibib}
\newcites{papers}{\mysidestyle Selected Papers}
\newcites{talks}{\mysidestyle Selected Talks}
\newcites{works}{\mysidestyle Selected Works}

\usepackage{etoolbox}
\BeforeBeginEnvironment{thebibliography}{%
  \let\origsection\section% save original definition of \section
  \let\section\subsection%  make \section behave like \subsection
}
\AfterEndEnvironment{thebibliography}{%
  \let\section\origsection% restore original definition of \section
}

\usepackage{fancyhdr}
\pagestyle{fancy}
\renewcommand{\headrulewidth}{0pt}
\newcommand{\COMMITHASH}{GITHUBCOMMITHASH}
\newcommand{\RUNNUMBER}{GITHUBRUNNUMBER}
\fancyhead{}
\fancyfoot{}
\lfoot{\hspace{-\sectionwidth}\footnotesize {Scott Phillips' Curriculum Vitae}}
\rfoot{\footnotesize Built {November 12, 2024}}

\linespread{1.0}

%______________________________________________________________________________________________________________________
\begin{document}
\name{\Large Scott Phillips}
\begin{resume}

%__________________________________________________________________________________________________________________
% Contact Information
\section{\mysidestyle Contact\\Information}

Scott Phillips\hfill \href{mailto:scjphill@ucsc.edu}{\faicon{envelope}~scjphill@ucsc.edu}
\vspace{0mm}\\\vspace{0mm}%
215 Felix, Santa Cruz, CA 95060\hfill  \href{https://www.linkedin.com/in/scott-phillips-438583176/}{\faicon{linkedin}~Scott Phillips}
\vspace{0mm}\\\vspace{0mm}%
(559) 360-7999\hfill \href{https://github.com/ScottieJPhillips}{\faicon{github}~ScottieJPhillips}
\vspace{0mm}\\\vspace{0mm}%
\vspace{-6.5mm}%

%__________________________________________________________________________________________________________________
% Research Interests
\section{\mysidestyle Research\\Interests}
{\small
High Energy Particle Physics, Electronics, Machine Learning, Data Analysis\\Computational Physics.
}
%__________________________________________________________________________________________________________________
% Education
\section{\mysidestyle Education}

\textbf{University of California Santa Cruz}, Santa Cruz, California \hfill \textbf{September 2023 -- June 2025}\\
\textsl{B.S. Physics}\\

\textbf{Fresno City College}, Fresno, California \hfill \textbf{August 2021 -- May 2023}\\
\textsl{A.S. Physical Science, A.S.-T Chemistry}
%
% \textbf{Florida Atlantic University} Boca Raton, Florida \hfill \textbf{Aug 2004 -- June 2008}\\
% \textsl{Engineering Scholars Program} -- dual-enrollment during high school\\[5mm]
%
% \textbf{Suncoast Community High School}, Riviera Beach, Florida \hfill \textbf{Aug 2004 -- Dec 2008}\\
% \textsl{Math, Science and Engineering Program}

%__________________________________________________________________________________________________________________
% Professional Experience
\section{\mysidestyle Experience}

\textbf{SCIPP}, University of California, Santa Cruz \hfill \textbf{November 2023 -- Present}\\
\textsl{Pixel Detector Project Assistant, ATLAS Experiment at CERN}
\begin{list2}
\item Ran thermal stress and power cycle tests on hybrid pixel sensors to assess long-term reliability.
\item Built a custom acrylic jig for sensor positioning and stability during testing.
\item Wrote and ran automated scripts to measure electrical performance with lab DAQ and power equipment.
\item Created a simple terminal-based UI for controlling test hardware and tracking chip temperature and power.
\item Assembled components in cleanroom environments, following ESD-safe and handling protocols.
\end{list2}


\textbf{SCIPP}, University of California Santa Cruz \hfill \textbf{October 2024 -- June 2025}\\
\textsl{\href{https://github.com/ScottieJPhillips/Gradient-Based-Learning-of-Photon-Selection-Cuts}{\faicon{github}~Gradient Based Learning of Photon Selection Cuts}: Cuts as Biases in Networks}
\begin{list2}
\item Built a custom neural network and loss function to optimize photon ID in ATLAS Monte Carlo data.
\item Applied gradient descent to tune selection cuts directly, improving signal efficiency.
\item Processed large datasets on SLURM-based HPC clusters at the University of Chicago.
\item Investigated interpretability methods to study how cut-based biases shape network learning.
\end{list2}

\textbf{Metiri}, Clovis, California \hfill \textbf{May 2022 -- Jan 2024}\\
\textsl{Technician/Analyst, Volatile Organic Analysis}
\begin{list2}
\item Operated GC-MS systems for detecting volatile and semi-volatile organic compounds.
\item Programmed SIM methods for targeted compound detection.
\item Interpreted chromatograms and spectra to quantify pollutant concentrations.
\item Performed QC and calibration within DoD and EPA guidelines.
\item Drafted reports and reviewed data for regulatory submission.
\end{list2}
%__________________________________________________________________________________________________________________

%__________________________________________________________________________________________________________________

% Honours and Awards
\section{\mysidestyle Honours and\\Awards}

UC Santa Cruz Deans Honor Award\\
Fresno City College Deans Honors Award\\
Clovis Community College Deans Honors Award\\

%__________________________________________________________________________________________________________________

% Languages, Programming, Skills

\section{\mysidestyle Technical \\Skills}
\textbf{Instrumentation}: Oscilloscope, power supply, multimeter, soldering (SMD + through-hole), DAQ systems, GC-MS, microscope work, vibration/thermal test jigs

\textbf{Software \& Libraries}: Python, C++, ROOT, TensorFlow, Keras, NumPy, SciPy, Matplotlib, Git

\textbf{CAD \& Hardware}: AutoCAD, 3D printing (FDM), PCB testing, TikZ circuit schematics

\textbf{Platforms}: Linux, GitHub/GitLab, SLURM HPC clusters

\end{resume}
\end{document}